\documentclass[a4paper, 12pt]{article}
\usepackage[margin=1in]{geometry}
\usepackage[utf8]{inputenc}
\usepackage{amsmath}
\usepackage{amssymb}

\title{Cauchy-Riemann Equations}
\author{}
\date{}

\begin{document}

\maketitle

Let
\begin{equation}
f(x,y)\equiv u(x,y)+iv(x,y),
\end{equation}
where
\begin{equation}
z \equiv x+iy,
\end{equation}
so
\begin{equation}
dz=dx+idy.
\end{equation}
The total derivative of $f$ with respect to $z$ is then
\begin{align}
    \frac{df}{dz}&=\frac{\partial f}{\partial x}\frac{\partial x}{\partial z}+\frac{\partial f}{\partial y}\frac{\partial y}{\partial z}\\
    &=\frac{1}{2}\left(\frac{\partial f}{\partial x} - i\frac{\partial f}{\partial y}\right).\label{eq:5}
\end{align}
In terms of $u$ and $v$, (\ref{eq:5}) becomes 
\begin{align}
\frac{df}{dz}&=\frac{1}{2}\left[\left(\frac{\partial u}{\partial x}+i\frac{\partial v}{\partial x}\right)-i\left(\frac{\partial u}{\partial y} + i\frac{\partial v}{\partial y}\right)\right]\\
&=\frac{1}{2}\left[\left(\frac{\partial u}{\partial x}+i\frac{\partial v}{\partial x}\right)+\left(-i\frac{\partial u}{\partial y} + \frac{\partial v}{\partial y}\right)\right].
\end{align}
Along the real, or $x$-axis, $\partial f/\partial y=0$, so
\begin{equation}\label{eq:8}
    \frac{\partial f}{\partial z} = \frac{1}{2}\left(\frac{\partial u}{\partial x} + i\frac{\partial v}{\partial x}\right).
\end{equation}
Along the imaginary, or $y$-axis, $\partial f/\partial x = 0$, so
\begin{equation}\label{eq:9}
    \frac{\partial f}{\partial z} = \frac{1}{2}\left(-i\frac{\partial u}{\partial y} + \frac{\partial v}{\partial y}\right).
\end{equation}
If $f$ is complex differentiable, then the value of the derivative must be the same for a given $dz$, regardless of its orientation. Therefore, (\ref{eq:8}) must equal (\ref{eq:9}), which requires that
\begin{equation}
    \frac{\partial u}{\partial x} = \frac{\partial v}{\partial y}
\end{equation}
and
\begin{equation}
    \frac{\partial v}{\partial x} = -\frac{\partial u}{\partial y}.
\end{equation}
These are known as the Cauchy-Riemann equations.\\
They lead to the conditions
\begin{align}
    \frac{\partial^2 u}{\partial x^2} &= -\frac{\partial^2 u}{\partial y^2}\\
    \frac{\partial^2 v}{\partial x^2} &= -\frac{\partial^2 v}{\partial y^2}.
\end{align}
The Cauchy-Riemann equations may be concisely written as
\begin{align}
    \frac{\partial f}{\partial \bar z} &= \frac{1}{2}\left[\frac{\partial f}{\partial x} + i\frac{\partial f}{\partial y}\right]\\
    &=\frac{1}{2}\left[\left(\frac{\partial u}{\partial x} + i\frac{\partial v}{\partial x}\right) + i\left(\frac{\partial u}{\partial y} + i\frac{\partial v}{\partial y} \right)\right]\\
    &=\frac{1}{2}\left[\left(\frac{\partial u}{\partial x} - \frac{\partial v}{\partial y}\right) + i\left(\frac{\partial u}{\partial y} + i\frac{\partial v}{\partial x} \right)\right]\\
    &= 0,
\end{align}
where $\bar z$ is the complex conjugate.\\
If $z = re^{i\theta}$, then the Cauchy-Riemann equations become
\begin{align}
    \frac{\partial u}{\partial r} &= \frac{1}{r}\frac{\partial v}{\partial \theta}\\
    \frac{1}{r}\frac{\partial u}{\partial \theta} &= -\frac{\partial v}{\partial r}
\end{align}
(Abramowitz and Stegun 1929, p. 17).\\
If $u$ and $v$ satisfy the Cauchy-Riemann equations, they aslo satisfy Laplace's equation in two dimensions, since
\begin{align}
    \frac{\partial^2 u}{\partial x^2}+\frac{\partial^2 u}{\partial y^2} &= \frac{\partial}{\partial x}\left(\frac{\partial v}{\partial y}\right) + \frac{\partial}{\partial y}\left(-\frac{\partial v}{\partial x}\right) = 0\\
    \frac{\partial^2 v}{\partial x^2}+\frac{\partial^2 v}{\partial y^2} &= \frac{\partial}{\partial x}\left(-\frac{\partial u}{\partial y}\right) + \frac{\partial}{\partial y}\left(\frac{\partial u}{\partial x}\right) = 0
\end{align}
By picking an arbitrary $f(z)$, solutions can be found which automatically satisfy the Cauchy-Riemann equations and Laplace's equation. This fact is used to use conformal mappings to find solutions to physical problems involving scalar potentials such as fluid flow and electrostatics.
\end{document}
