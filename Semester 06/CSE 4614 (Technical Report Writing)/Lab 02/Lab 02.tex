% additional parameters can be used with the document class to define things like the page and font size
\documentclass[a4paper, 12pt]{article}
\usepackage[utf8]{inputenc}
\usepackage{graphicx}   % required for images
\usepackage{subcaption} % required for sub-images
\usepackage[margin=1in]{geometry}
% the geometry package allows us to specify margins
% we can also specify specific margins as below
% \usepackage[top=1in, bottom=1in]{geometry}
% when binding the paper (e.g. for a thesis), we can use the following command to add the required margins on the correct side of each page
% \usepackage[bindingoffset=1cm]{geometry}

\title{Title of the Paper}
\author{My Name}
\date{December 04, 2021}

\begin{document}

\maketitle

Please refer to the comments of the code for further information, especially in the header.

\bigskip

We can create a table of contents as below.
\tableofcontents

\bigskip

We can also create a list of all the figures used in the paper as below.
\listoffigures

\newpage

\begin{abstract}
    We can create a separate abstract section like this.
\end{abstract}

\section{Section 1}
A reference has been used here. It requires specifying a label for the section, which can be used in the reference.

\label{sec:sec1}
This is Section \ref{sec:sec1}.

This reference will automatically reflect changes in the section title.

\subsection{Subsection 1}
This is subsection 1.

\subsubsection{SubSubSection}
This is subsubsection 1.

\section{Bullet Points}
We can create a bulleted list like this:
\begin{itemize}
    \item Item 1
    \begin{itemize}
        \item Item 1.1
    \end{itemize}
\end{itemize}

We can also have custom bullet points.
\begin{itemize}
    \item[*] Item 1
\end{itemize}

We can create a numbered list like this:
\section{Numbered Lists}
\begin{enumerate}
    \item Item 1
    \begin{enumerate}
        \item Item 1.1
    \end{enumerate}
    \item Item 2
\end{enumerate}

\section{Images}
We can insert an image like this. Note that in Overleaf, we must upload the image first. Also note that this requires the \textbf{graphicsx} package.

\newpage

\begin{figure}
    \centering
    \includegraphics{image1.jpg}
    \caption{Initial Image}
\end{figure}

The position of the image on the page can be adjusted using extra parameters. \textbf{h} refers to this exact position, \textbf{t} refers to the top of the page and \textbf{b} refers to the bottom of the page. By default, images go to the top of the page.

\begin{figure}[h]
    \centering
    \includegraphics{image1.jpg}
    \caption{Image Position Example}
\end{figure}

We can also specify the exact size of the image. The scale parameter scales the image, the width parameter specifies a specific width and the height parameter specifies a specific height.

\begin{figure}[h]
    \centering
    \includegraphics[scale=0.5]{image1.jpg}
    \caption{Scaled Down Image Example}
\end{figure}

We can also tell the image to be scale based on the width of the text area.

\begin{figure}
    \centering
    \includegraphics[width=0.8\textwidth]{image1.jpg}
    \caption{Image Scaled Based on Text Width Example}
\end{figure}

\newpage

We can also use labels to make references to images.

\begin{figure}[h]
    \centering
    \includegraphics[scale=0.5]{image1.jpg}
    \caption{Image Reference Example}
    \label{fig:imageref}
\end{figure}

This is a reference to figure \ref{fig:imageref}.
By convention, the label for figure references is \textbf{fig:referencename}.

We do not always use the format \textbf{figure 5} though. We may want to use \textbf{fig. 5} instead. In fact, many publishers require this. We can specify which one to use like this:

This is a reference to \figurename~\ref{fig:imageref}.

The tilde symbol has been used to add a space.

We can adjust the horizontal position of the image like this:

\begin{figure}[h]
    \flushright
    \includegraphics[scale=0.5]{image1.jpg}
    \caption{Image Position Example}
\end{figure}

We could also have used flushleft.

And finally, we can have multiple images within an image, each with a separate caption, like this. Note that this requires the \textbf{subcaption} package.

\begin{figure}
    \centering
    \subfloat[Subfigure 1]{
        \includegraphics[width=0.4\textwidth]{image1.jpg}
    }
    \subfloat[Subfigure 2]{
        \includegraphics[width=0.4\textwidth]{image1.jpg}
    }
    \caption{Subfigure Example}
\end{figure}

\newpage

We can also put the subfigures vertically like this:

\begin{figure}[h]
    \centering
    \subfloat[Subfigure 1]{
        \includegraphics[width=0.4\textwidth]{image1.jpg}
    }
    \subfloat[Subfigure 2]{
        \includegraphics[width=0.4\textwidth]{image1.jpg}
    }
    \hspace{1cm}
    \subfloat[Subfigure 3]{
    \includegraphics[width=0.4\textwidth]{image1.jpg}
    }
    \caption{Vertical Subfigures Example}
    
\end{figure}

\end{document}