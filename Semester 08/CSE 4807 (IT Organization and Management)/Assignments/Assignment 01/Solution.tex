\documentclass{article}
\usepackage[english]{babel}
\usepackage[a4paper]{geometry}
\usepackage{amsmath}
\usepackage{graphicx}
\usepackage[colorlinks=true, allcolors=blue]{hyperref}

\usepackage{fancyhdr}
\pagestyle{fancy}
\fancyhead{} % clear all header fields
\fancyhead[L]{CSE 4807: IT Organization and Management} % header fields
\fancyhead[R]{Assignment 01} % header fields
\renewcommand{\headrulewidth}{0.5pt} % no line in header area
\fancyfoot{} % clear all footer fields
\fancyfoot[R]{\thepage} % page number


\title{Case Studies on Software Firm Challenges\\CSE 4807}
\author{Alvi Aveen Khan - 180041229}
\date{\today}

\begin{document}
\maketitle

\section{Case Study 01: Difficulties Caused by Inflation}
\subsection{Analysis}
The recent economic inflation has led to a two-fold burden for the business. There has been an increased fluctuation in supply and demand, which has led to financial difficulties for the business. If not addressed swiftly and appropriately, this can lead to the business going bankrupt due to being unable to meet its costs.

One such cost is the salary expectations of its employees. The financial difficulties can result in the business being unable to pay out the agreed-upon salaries to its employees. This is, of course, unlikely to be well received by the employees, especially when they are also likely facing the effects of the economic inflation. If the issue is egregious enough, the company may have trouble retaining its employees for long, which will only exacerbate its problems.

\subsection{Business Needs}
The business must meet the salary expectations of its employees in order to retain them. At the same time, it must survive the fluctuations in supply and demand, which are causing additional financial difficulties.

\subsection{Recommendations}
To address the issue of fluctuating supply and demand, the company must identify areas where more resources than necessary are being spent and readjust the priorities of the tasks. This can include redistributing employees working on different projects to minimize the amount of time spent idle. The optimal way to handle this would be to go one step further and create a working environment where employees can be redistributed at short notice since that will best address the fluctuations in demand. Being able to implement this appropriately will lead to a stable flow of income, which will in turn address the issue of meeting the salary expectations of employees.

\subsection{Implementation Plan}
The first step in implementing the recommendations is to spend some time analysing which projects are meeting, ahead of, or falling behind on their deadlines. This should provide the primary data required to go ahead with the initial redistribution of employees.

The second step is a long-term one since it involves training the employees to have at least the basic knowledge required to work on multiple different projects. Having this training will allow the employees to later switch to other projects if further redistribution is required.

\subsection{Evaluation}
To evaluate the outcome of the proposed changes, the analysis of met and overdue deadlines must be conducted again. The first analysis should take place after the initial redistribution since it will provide justification to continue using the proposed method. Further analysis can be conducted later in the future after more redistributions have taken place. This will provide a judgement of whether the repeated redistributions are effective in handling the fluctuations. It is important to ensure that the repeated redistributions do not harm productivity to an unacceptable extent.

\newpage

\section{Case Study 02: Foreign Competition}
\subsection{Analysis}
The business has recently begun facing strong competition from foreign companies. Being unable to impress clients and essentially outbid their competition will result in the business losing customers. Poor customer retention can have a snowball effect, since the loss of customers will reflect negatively on the business, driving away potential future customers.

\subsection{Business Needs}
The business must find a way to outperform their competition, either in terms of the quality of their software, the breadth of their expertise, or the cost of their services.

\subsection{Recommendations}
The majority of the solutions to the problems depend on the amount of monetary resources available to the business. Both software quality and expertise breadth can be increased by training the current employees and hiring new employees who are more experienced in the field. Both of these can be costly endeavours.

If the business chooses to outbid their competition by providing cheaper services, they must do so while still maintaining similar quality to their competition. Otherwise, customers, especially high-value ones, are unlikely to pick the cheaper option given that it is also worse. Of course, being able to provide quality service for a cheaper price means having employees who are willing to work for less. This may not be a possibility depending on the location of the business.

\subsection{Implementation Plan}
The option of offering a cheaper price at the cost of lower employee wages does not require much explanation, so the implementation section will concentrate on the training and hiring options.

Initially, the company should conduct an investigation into three things: what services are being provided by their competitors, which of those services are not provided by the business, and the skills that their employees possess related to the services not yet provided. This will bring out the areas in which there is a lack of experience amongst the employees, which can be used to determine what training to provide to current employees and what type of new employees to look for.

\subsection{Evaluation}
The evaluation of the implemented methods will be related to customer retention rates and sales revenue. Increased customer retention is proof that the business is losing fewer customers to their competitors. Increased sales revenue, as well as the margin of the increase, will provide information regarding the retention of high-value customers. If the business is particularly good at retaining high-value customers, it indicates that the quality, breadth, and cost of their services are more competitive than the rest of the market.

\section{Case Study 03: Customer Retention}
\subsection{Analysis}
The business has recently begun losing customers for unknown reasons. The loss of customers will of course lead to immediate issues due to decreased profits, but it can also cause long-term issues for the business since the lost customers are unlikely to recommend the business to their contacts, which means that potential future customers are being lost as well.

There are a variety of possible reasons for the change in customer loyalty. Being unable to keep up with competitors and poor customer service tend to be the most common ones.

\subsection{Business Needs}
The business must identify the cause behind the poor customer retention and find ways of addressing the issues.

\subsection{Recommendations}
To remediate the issue, the business must conduct extensive surveys. Customers who choose to leave the business can be surveyed on their reasons, and existing customers could be surveyed for areas where the business could improve.

If the issue is related to being unable to compete with the rest of the market, the recommendations for this situation have already been discussed above.

If the issue is related to poor customer service, the business must immediately begin training their employees on how to present a more appealing attitude towards customers, improve response times, and provide more accurate and easily understandable solutions to their issues.

Other generic methods that can be used to retain customers which do not consider the underlying causes for the loss of customers in the first place include providing discounts, investing in advertising, etc.

\subsection{Implementation Plan}
The initial survey of customers will take some time, but given that the process is implemented thoroughly and in a manner such that customers are likely to give genuine answers, the results of this stage should prove extremely useful. It is important to keep in mind that the survey must be conducted in a user-friendly and easily answerable manner so as to decrease the amount of annoyance caused to the customers. This is particularly important for the group of customers who are leaving the business because they are the least likely to want to assist the business in any way.

Once the data from the survey has been collected, it can be determined whether the issue is with providing worse services than competitors or with poor customer interactions. Since the issues of being unable to compete with competition have already been discussed, the latter issue is only being addressed here.

To improve customer interactions, the business must train employees who directly interact with customers. This involves workshops, video tutorials, and practice sessions. The employees must be able to remain friendly under stress and also have enough knowledge about how the services provided by the business work to be able to answer any questions quickly and correctly. This stage can be particularly difficult from a monetary perspective, since not only are the employees unable to work during the time they are involved with the training, but the training itself may also cause additional expenses if external assistance is required.

\subsection{Evaluation}
Customer retention is a well-established evaluation metric and can be directly applied to this scenario. Improved customer retention will, of course, indicate that the implemented methods are working. Further surveys can also be conducted to gather data about the opinions of customers regarding the changes.

\end{document}